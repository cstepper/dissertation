%************************************************
\chapter{Introduction}
\label{chp:Introduction}
%************************************************

Digital aerial photogrammetry experienced increased attention during the last decade. 


\section{State of the Art}\label{sec:State}

Literature review on remote sensing applications in forestry. Historic development of aerial imagery for use in forest inventory and management planning.

Early developments of aerial photogrammetry for height retrieval - theoretical framework

3d remote sensing technologies: Airborne laser scanning as method of choice.

Here we want to cite \cite{Stepper.2015b}




\section{Research Objectives}\label{sec:Objectives}

The research presented in this thesis focussed on the following objectives:

\begin{enumerate}
	
	\item Are digital aerial images acquired within the standardized administrative aerial surveys suitable to computed dense image-based point clouds or digital surface models, that characterize the forests' surface with a sufficient level of detail?
	
	\item Can Semi-global matching point clouds, normalized to heights above ground using ALS-based DTMs, be used to model key forest inventory attributes, e.g. gross volume?
	
	\item Are repeated aerial image acquisitions and the derived CHMs capable to assess canopy height changes is a complex temperate forest?
	
\end{enumerate}





