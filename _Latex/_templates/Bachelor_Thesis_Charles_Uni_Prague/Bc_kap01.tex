%%%
%%% BACHELOR'S THESIS TEMPLATE - ENGLISH
%%%  
%%%  * the first chapter
%%%
%%%  AUTHORS:  Arnost Komarek (komarek@karlin.mff.cuni.cz), 2011
%%%            Michal Kulich (kulich@karlin.mff.cuni.cz), 2013
%%%
%%%  LAST UPDATED: 20130318
%%%  

\chapter{Typesetting of mathematical text}

This chapter demonstrates a few examples of mathematical text typesetting.

%%%%% ===============================================================================
\section{Basic examples}


A number in the mathematical mode with decimal point: $\pi \doteq 3.141\,592\,653\,589$.

Test on a 5\% level, 95\% confidence interval.

We have $\var(X) = \E X^2 - \bigl(\E X \bigr)^2$.

\section{Mathematical expressions}
Let
\[
\mathbb{X} = \begin{pmatrix}
      \T{\bm x_1} \\
      \vdots \\
      \T{\bm x_n}
      \end{pmatrix}.
\]
Note the period after the matrix. The mathematical text is a part of
the sentence grammar and requires standard punctuation. Expressions
that will be referenced should be numbered:
\begin{equation}\label{eq01:Xmat}
\mathbb{X} = \begin{pmatrix}
      \T{\bm x_1} \\
      \vdots \\
      \T{\bm x_n}
      \end{pmatrix}.
\end{equation}
Expression \eqref{eq01:Xmat} defines matrix $\mathbb{X}$. To achieve
better readability and organization, it is advised to number only those
expressions that are referenced. Do not number all displayed
expressions automatically.

Aligning of expressions into several columns:
\begin{alignat*}{3}
S(t) &= \pr(T > t),    &\qquad t&>0       &\qquad&\text{ (right-continuous),}\\
F(t) &= \pr(T \leq t), &\qquad t&>0       &\qquad&\text{ (right-continuous).}
\end{alignat*}

Two binded equations:
\begin{equation}\label{eq01:FS}
\left.
\begin{aligned}
S(t) &= \pr(T > t) \\[1ex]
F(t) &= \pr(T \leq t)
\end{aligned}
\right\}
\quad t>0 \qquad \text{(right-continuous).}
\end{equation}

Two centered unnumbered equations:
\begin{gather*}
\bm Y = \mathbb{X}\bm\beta + \bm\varepsilon, \\[1ex]
\mathbb{X} = \begin{pmatrix} 1 & \T{\bm x_1} \\ \vdots & \vdots \\ 1 &
  \T{\bm x_n} \end{pmatrix}.
\end{gather*}
Two centered numbered equations:
\begin{gather}
\bm Y = \mathbb{X}\bm\beta + \bm\varepsilon, \label{eq02:Y}\\[1ex]
\mathbb{X} = \begin{pmatrix} 1 & \T{\bm x_1} \label{eq03:X}\\ \vdots & \vdots \\ 1 &
  \T{\bm x_n} \end{pmatrix}.
\end{gather}

Definition organized by cases:
\[
P_{r-j}=
\begin{cases}
0 & \text{if $r-j$ is odd},\\
r!\,(-1)^{(r-j)/2} & \text{if $r-j$ is even}.
\end{cases}
\]
Note the use of punctuation in the equation. Commas and periods are
placed according to the standard English language rules.

\begin{align}
x& = y_1-y_2+y_3-y_5+y_8-\dots && \text{by \eqref{eq02:Y}} \nonumber\\
& = y'\circ y^* && \text{by \eqref{eq03:X}} \nonumber\\
& = y(0) y' && \text {by Axiom 1.}
\end{align}

Two unnumbered aligned equations:
\begin{align*}
L(\bm\theta) &= \prod_{i=1}^n f_i(y_i;\,\bm\theta), \\
\ell(\bm\theta) &= \log\bigl\{L(\bm\theta)\bigr\} = 
\sum_{i=1}^n \log\bigl\{f_i(y_i;\,\bm\theta)\bigr\}.
\end{align*}
Two aligned equations, the first numbered:
\begin{align}
L(\bm\theta) &= \prod_{i=1}^n f_i(y_i;\,\bm\theta), \label{eq01:L} \\
\ell(\bm\theta) &= \log\bigl\{L(\bm\theta)\bigr\} = 
\sum_{i=1}^n \log\bigl\{f_i(y_i;\,\bm\theta)\bigr\}. \nonumber
\end{align}

Two-line equation, the first line aligned left, the second line
aligned right, unnumbered:
\begin{multline*}
\ell(\mu,\,\sigma^2) = \log\bigl\{L(\mu,\,\sigma^2)\bigr\} = 
\sum_{i=1}^n \log\bigl\{f_i(y_i;\,\mu,\,\sigma^2)\bigr\} \\
  = -\,\frac{n}{2}\,\log(2\pi\sigma^2) \,-\, 
\frac{1}{2\sigma^2}\sum_{i=1}^n\,(y_i - \mu)^2. 
\end{multline*}

Two-line equation, aligned to $=$, numbered in the middle:
\begin{equation}\label{eq01:ell}
\begin{split}
\ell(\mu,\,\sigma^2) &= \log\bigl\{L(\mu,\,\sigma^2)\bigr\} = 
\sum_{i=1}^n \log\bigl\{f(y_i;\,\mu,\,\sigma^2)\bigr\} \\
& = -\,\frac{n}{2}\,\log(2\pi\sigma^2) \,-\, 
\frac{1}{2\sigma^2}\sum_{i=1}^n\,(y_i - \mu)^2. 
\end{split}
\end{equation}


%%%%% ===============================================================================
\section{Definitions, theorems, proofs, \dots}

Constructions like definitions, theorems, examples, \dots, should be
separated from the surrounding  text and (usually) numbered, with the
use of cross-references. For each of these constructions, the main
source file (\texttt{BcPrace.tex}) should define an environment
providing its visual separation from the surrounding text and
numbering/cross-referencing features. 

\begin{definice}\label{def01:1}
  Let random variables $X_1,\dots,X_n$ be defined on the same
  probability space $(\Omega,\,\mathcal{A},\,\pr)$. Then the vector
  $\bm X = \T{(X_1,\dots,X_n)}$ is called a \emph{random
    vector}.
\end{definice}

\begin{definice}[random vector]\label{def01:2}
  Let random variables $X_1,\dots,X_n$ be defined on the same
  probability space $(\Omega,\,\mathcal{A},\,\pr)$. Then the vector
  $\bm X = \T{(X_1,\dots,X_n)}$ is called a \emph{random
    vector}.
\end{definice}
Definition~\ref{def01:1} demonstrates an environment without a
subtitle, definition~\ref{def01:2} includes a subtitle.

\begin{veta}\label{veta01:1}
  The random vector $\bm X$ is a measurable mapping of the space 
  $(\Omega,\,\mathcal{A},\,\pr)$ to $(\R_n,\,\mathcal{B}_n)$.
\end{veta}

\begin{lemma}[\citealp{Andel07}, str. 29]\label{veta01:2}
  The random vector $\bm X$ is a measurable mapping of the space 
  $(\Omega,\,\mathcal{A},\,\pr)$ to $(\R_n,\,\mathcal{B}_n)$.
\end{lemma}
\begin{dukaz}
  The detailed steps of the proof are described in the book \citet[p.
  29]{Andel07}. 
\end{dukaz}
Theorem~\ref{veta01:1} demonstrates an environment without a subtitle,
lemma~\ref{veta01:2} includes a reference in the subtitle. Lemmas were
defined in the main file so that they share the same numbering with
theorems.



