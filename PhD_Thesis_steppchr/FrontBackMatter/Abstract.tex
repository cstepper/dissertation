%*******************************************************
% Abstract + Zusammenfassung
%*******************************************************

\pdfbookmark[1]{Abstract}{Abstract}
\begingroup
\let\clearpage\relax
\let\cleardoublepage\relax
\let\cleardoublepage\relax

\chapter*{Abstract}
Information about canopy height and spatial structure is essential for various tasks in forest planning and management. Retrieving this information from remote sensing data has been focus in research as this technique opens the possibility to assess large forested areas. 
The work presented in this thesis focussed on utilizing digital aerial imagery from airborne platforms to generate accurate measurements of canopy height by means of image matching algorithms. Together with ground information, necessary for model calibration, different forest inventory attributes can be assessed \dots

\vfill

\begin{otherlanguage}{ngerman}
	\pdfbookmark[1]{Zusammenfassung}{Zusammenfassung}
	
\chapter*{Zusammenfassung}
Informationen über die Bestandeshöhe und die räumliche Verteilung der Bäume in einem Bestand \dots 

\end{otherlanguage}



\endgroup			

\vfill

