%************************************************
\chapter{Conclusion}
\label{chp:Conclusion}
%************************************************

% A short conclusion to summarize the presented methods, results, and discussion.

The presented methods and achieved results from the different studies approved the high potential
of \ac{DAP} for use in forest inventory and monitoring tasks.
In the studies, image-based height data were employed in combination with different field-measured information sources
(\aclp{MFI}, \acl{NFI}) on a range of spatial scales.
Additionally, due to the lower acquisition costs compared to \ac{ALS}, and the high recurrence frequency of the 
administrative aerial surveys, airphotos and derived \ac{DAP} products
can provide an important bridge between the operational, tactical, and strategic forest management levels. 

The studies confirmed that \ac{DAP}-based height information can be used in conjunction with ground-based measurements 
to establish predictive models for a set of forest inventory attributes \parencites[see][]{Stepper.2015b, White.2015, Straub.2016, Stepper.2016}.
Similar to \ac{ALS}, the most predictive canopy metrics derived from the \ac{DAP}-based measurements were height metrics, 
which are most appropriate to describe the canopy \emph{growing space}. 
It could be demonstrated, that the intra-metric correlation is higher for the image-based data compared to the \ac{ALS} data \parencite[see][]{White.2015},
indicating that it is sufficient for modelling to work with a subset of the computed metrics. 
In \textcite{Stepper.2015b} we could show, that \dots



\begin{itemize}
	\item Baumartenklassifikation ?!?
	\item UAV?
\end{itemize}