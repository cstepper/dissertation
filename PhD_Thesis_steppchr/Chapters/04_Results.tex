%************************************************
\chapter{Results}
\label{chp:Results}
%************************************************

The results of the conducted research studies were presented to scholars in different scientific papers 
published in peer-reviewed journals (see list of publications on page~\pageref{chp:Publications}). 
In the following, the main findings of the respective papers are summarized (section~\ref{sec:overallResults}).
Afterwards, the results of each paper are given in detail and my personal contribution to each paper is highlighted (section~\ref{sec:detailedResults}). 

\section{Overall results of the presented work}\label{sec:overallResults}

In Table~\ref{tab:ResultsOverview}, a very general overview of the achieved results according to the overall aims of this work (see section~\ref{sec:Objectives}) is given.
The first aim, \ie, the examination of image-based height measurements derived from stereo remote sensing data sets 
for the description of forest canopy surfaces, was pursued in all studies. 

  

\begin{threeparttable}[t]
	\myfloatalign
	\caption[Overview of research aims and achieved results covered by the respective studies.]{Overview of research aims and achieved results covered by the respective studies.}
		\label{tab:ResultsOverview}
	\small
	\begin{tabularx}{1\textwidth}{p{2.4cm}XXX}
		\toprule
		\tableheadline{Studies} & 
		\spacedlowsmallcaps{Aim 1:} Are digital aerial images and \ac{DAP} suitable to compute image-based outputs to characterize forest canopy surfaces? & 
		\spacedlowsmallcaps{Aim 2:} Can image-based point clouds or raster-based surface models be used to model forest inventory attributes?& 
		\spacedlowsmallcaps{Aim 3:} Are \acp{CHM} derived from repeated airphoto acquisitons capable to assess canopy height changes over time?\\
		\midrule
		Stepper et al. \linebreak \emph{Can. J. For. Res.} (2015) & \centering\Large\checkmark & \centering\Large\checkmark &  \\ 
		Straub\,\&\,Stepper \linebreak \emph{Photogramm. Fernerkund. Geoinf.} (in~press) &\centering\Large\checkmark & \centering\Large\checkmark &  \\ 
		Stepper et al. \linebreak \emph{Scand. J. For. Res.} (submitted) & \centering\Large\checkmark & \centering\Large\checkmark &  \\ 
		White et al. \linebreak \emph{Forests} (2015) & \centering\Large\checkmark & \centering\Large\checkmark &  \\ 
		Immitzer et al. \linebreak \emph{Forest Ecol. Manag.} (2016) & \centering\Large\checkmark\small\tnote{a} & \centering\Large\checkmark\small\tnote{b} &  \\ 
		Stepper et al. \linebreak \emph{Forestry} (2015) & \centering\Large\checkmark && \centering\Large\checkmark \tabularnewline
		\bottomrule
	\end{tabularx}
		\begin{tablenotes}
			\item[a] \footnotesize{Stereo satellite image data were used in this study for image matching and consecutive computation of canopy surface heights.}
			\item[b] \footnotesize{In contrast to the other studies, \ac{ACS}-based field data from \ac{NFI} measurements were used for model training in this study.}
		\end{tablenotes}	
\end{threeparttable}



	
%\begin{savenotes}	
%\begin{table}[htb]
%	\myfloatalign
%	\caption[Radii and DBH thresholds for the concentric sampling circles.]{Radii and \ac{DBH} thresholds for the concentric sampling circles
%		assembling a ground plot in typical \ac{BaySF} management inventories.}
%	\label{tab:OverallResults}
%	\small
%		\begin{tabularx}{1\textwidth}{p{2.3cm}XXX}
%			\toprule
%			\tableheadline{Studies} & 
%			\spacedlowsmallcaps{Aim 1:} Are digital aerial images and \ac{DAP} suitable to compute image-based point clouds to characterize forest canopy surfaces? & 
%			\spacedlowsmallcaps{Aim 2:} Can \ac{SGM} point clouds be used to model forest inventory attributes?& 
%			\spacedlowsmallcaps{Aim 3:} Are \acp{CHM} derived from repeated airphoto acquisitons capable to assess canopy height changes?\\
%			\midrule
%			 Stepper et al. \linebreak \emph{Can. J. For. Res.} \linebreak (2015) & 
%			 Dense image-based point clouds (avg. density $\>$ 11.6\,points$\cdot$m\textsuperscript{2}) were calculated by means of \ac{SGM}-- describing the canopy surface of a broadleaf-dominated forest. &
%			 \ac{RF} prediction models for gross volume were established and validated at the plot and stand levels (minimum RMSEs of $30.92\%$ and $13.87\%$, resp.).  & 
%			  \\
% 			 Straub\,\&\,Stepper \linebreak \emph{Photogramm.\:Ferner-kund.\:Geoinf.} \linebreak (in press) &
% 			 \acp{CHM} for a beech- and a spruce-domintated forest were computed from stereo airphotos using \ac{SGM}.& 
% 			 Prediction models for a set of five different forest inventory attributes -- $N$, $G$, $QMD$, $V$, and $H_L$\footnote{$N$: stem number, $G$: basal area,
% 			 	 $QMD$: quadratic mean diameter, $V$: volume, $H_L$: Lorey's mean height} -- were evaluated at the plot level;
% 			  achieved RMSEs were high for $N$, intermediate for $G$, $QMD$, $V$, and low for $H_L$.&
% 			  \\
%  			 Stepper et al. \linebreak \emph{Scand. J. For. Res.} \linebreak (submitted) &
%  			 Dense point clouds for a beech- and a spruce-domintated forest were computed from stereo airphotos using \ac{SGM}.& 
%  			 Prediction models for $d_{100}$, $h_{100}$\footnote{ $d_{100}$: quadratic mean diameter of the 100 largest trees, 
%  			 	$h_{100}$: basal area weighted height of the 100 largest trees}, and $V$ were trained for beech- and spruce-dominated forests;
%  			 spatial model transfer to nearby forests revealed \dots 
%  			 &
%  			 \\
%
% 			  
%%			 \\
%%			 
%%			 \adjustbox{valign=t}{\begin{tikzpicture}[baseline = 0]
%%			 \draw[gray,very thin] (0,0) -- (1.3,-0.5) -- (2.6,0) -- (2.6,2) -- (1.3,1.5) -- (0,2) -- cycle;
%%			 \node[align = center] at (1.3,0.75) {Stepper et al.\\ \emph{Can. J. For. Res.} \\ (2015)};
%%			 \end{tikzpicture}} &
%%			 a & b & c \\
%%			 
%%			  \adjustbox{valign=t}{\begin{tikzpicture}[baseline = 0]
%%			 \filldraw[fill=black!10!white, draw=black] (0,0) -- (1.3,-0.5) -- (2.6,0) -- (2.6,2) -- (1.3,1.5) -- (0,2) -- cycle;
%%			 \node[align = center] at (1.3,0.75) {Straub \& Stepper\\ \emph{PFG} \\ (in press)};
%%			 \end{tikzpicture}} &
%%			  a & b & c \\
%			\bottomrule
%		\end{tabularx}
%\end{table}
%\end{savenotes}
%	
%\begin{table}[htb]
%	\myfloatalign
%	\caption[Radii and DBH thresholds for the concentric sampling circles.]{Radii and \ac{DBH} thresholds for the concentric sampling circles
%		assembling a ground plot in typical \ac{BaySF} management inventories.}
%	\label{tab:OverallResults}
%	\small
%	\begin{tabularx}{.8\textwidth}{lXXX}
%		\toprule
%		\spacedlowsmallcaps{Study} & \spacedlowsmallcaps{DAP to characterize} \linebreak \footnotesize{[m]} & \spacedlowsmallcaps{model attri} \linebreak \footnotesize{[m\textsuperscript{2}]} & \spacedlowsmallcaps{asseisH} \linebreak \footnotesize{[cm]}  \\ 
%		\midrule
%		\begin{tikzpicture}
%		\draw[step=0.5cm,gray,very thin] (0,0) grid (4,4)
%		\draw (0,) -- (1,0.5) -- (2,1) -- (2,3) -- (1,2.5) -- (0,3) -- cycle; 
%		\node at (1,1) {\emph{Stepper et al 2915 sfaksöjgkljdfögjöadkjöf}};
%		\end{tikzpicture}
%		 & 2.82 & 25 & $<$\,12  \\ 
%		\emph{2} &  6.31 &  125 & 12--29.9 \\
%		\emph{3}  & 12.62 &  500 & $\geq$\,30  \\ 
%		\bottomrule
%	\end{tabularx}
%\end{table}
%	
%	
%	\begin{tikzpicture}
%	\draw (0,0) -- (1,-0.5) -- (2,0) -- (2,2) -- (1,1.5) -- (0,2) -- cycle; 
%	\node at (1,1) {\emph{Stepper et al 2915 sfaksöjgkljdfögjöadkjöf}};
%	
%	[
%	decoration={shape backgrounds,shape size=.5cm,shape=signal},
%	signal from=west, signal to=east,
%	paint/.style={decorate, draw=#1!50!black, fill=#1!50}]
%	\draw [help lines] grid (3,2);
%	\draw [paint=red, decoration={shape sep=.5cm}]
%	(0,2) -- (3,2);
%	\draw [paint=green, decoration={shape sep={1cm, between centers}}]
%	(0,1) -- (3,1);
%	\draw [paint=blue, decoration={shape sep={1cm, between borders}}]
%	(0,0) -- (3,0);
%	\end{tikzpicture}

\section[Detailed description of the results]{Detailed description of the results per article}\label{sec:detailedResults}

\subsection[Stepper et al. \emph{Can. J. For. Res.} (2015)]
		{Stepper et al.\; 
		\emph{Can. J. For. Res.}
		(2015) }
\label{sec:pub.CanJForRes2015}

\begin{description}
	\item[Title] Using semi-global matching point clouds to estimate growing stock at the plot and stand levels: 
		application for a broadleaf-dominated forest in central Europe
	\item[Authors] \emph{Christoph Stepper}, Christoph Straub, and Hans Pretzsch
	\item[Journal] Canadian Journal of Forest Research (Publisher: National Research Council Canada, NRC Research Press, \textsc{issn}: 0045-5067, 
		Journal Impact Factor, \emph{2014}: 1.683)
	\item[Abstract] Dense image-based point clouds have great potential to accurately assess forest attributes such as growing stock. The
		objective of this study was to combine height and spectral information obtained from UltraCamXp stereo images to model the
		growing stock in a highly structured broadleaf-dominated forest ($77.5\,km^2$) in southern Germany. We used semi-global matching
		(SGM) to generate a dense point cloud and subtracted elevation values obtained from airborne laser scanner (ALS) data to
		compute canopy height. Sixty-seven explanatory variables were derived from the point cloud and an orthoimage for use in the
		model. Two different approaches --- the linear regression model (lm) and the random forests model (rf) --- were tested. We
		investigated the impact that varying amounts of training data had on model performance. Plot data from a previously acquired
		set of 1875 inventory plots was systematically eliminated to form three progressively less dense subsets of 937, 461, and
		226 inventory plots. Model evaluation at the plot level (size: $500\,m^2$) yielded relative root mean squared errors (RMSEs)
		ranging from $31.27\%$ to $35.61\%$ for lm and from $30.92\%$ to $36.02\%$ for rf. At the stand level (mean stand size: $32\,ha$), RMSEs
		from $14.76\%$ to $15.73\%$ for lm and from $13.87\%$ to $14.99\%$ for rf were achieved. Therefore, similar results were obtained 
		from both modeling approaches. The reduction in the number of inventory plots did not considerably affect the precision. 
		Our findings underline the potential for aerial stereo imagery in combination with ALS-based terrain heights to support forest inventory and management.
	\item[Contribution] \myName and Christoph Straub developed the con-cept for the study and \myName designed the experimental setting. 
		Hans Pretzsch gave helpful comments on the study structure and the relevance of the research.
	
		\myName processed all ground inventory and remote sensing data. He performed the photogrammetric working steps
		to gain the image-based canopy height information and the orthoimage for the complete test site. 
		In this study, we scrutinized both height and spectral variables as predictor variables for modelling gross volume.
		\myName computed the point-cloud variables and Christoph Straub computed the variables extracted from the gridded \ac{CHM} and the orthoimage.
		\myName performed the area-based modelling workflow testing both regression approaches and applied the models for wall-to-wall mapping. 
		He conducted the model evaluation at the stand level and compiled all results. \myName and Christoph Straub interpreted the model outcomes
		and formulated the findings of this research.
	
		\myName prepared the manuscript including all figures and tables in close collaboration with Christoph Straub.
		Hans Pretzsch gave valuable advice for the interpretation of the results, for the discussion of the findings 
		and the composition of the manuscript. All authors contributed to the critical revisions of the article.
	
\end{description}


\subsection[Straub \& Stepper \emph{Photogramm.\:Fernerkund.\:Geoinf.} (in press)]
{Straub \& Stepper\; 
	\emph{Photogramm.\:Fernerkund.\:Geoinf.}
	(in press) }
\label{sec:pub.PFG2016}

\begin{description}
	\item[Title] Using digital aerial photogrammetry and the Random Forest approach to model forest inventory attributes
		in beech- and spruce-dominated central European forests
	\item[Authors] Christoph Straub and \emph{Christoph Stepper}
	\item[Journal] Photogrammetrie--Fernerkundung--Geoinformation, \emph{PFG} \\ (Publisher: Schweizerbart Science Publishers, \textsc{issn}: 1432-8364, 
		Journal Impact Factor, \emph{2014}: 0.733)
	\item[Abstract] Surface models generated using dense image matching of aerial photographs have 
		potential for use in the area-based prediction of forest inventory attributes. Few studies have
		examined the impact of forest type on the performance of models used to predict forest attributes.
		Moreover, with regard to central European forests, little is known about how accurately attributes
		other than volume and basal area can be estimated using image-based surface models. Thus, in this
		study, we assessed the accuracy of such estimates for five forest attributes --- stem density $N$, basal
		area $G$, quadratic mean diameter $QMD$, volume $V$, and Lorey’s mean height $H_L$ --- for a beech- and a
		spruce-dominated forest in northern Bavaria, Germany. These estimates were made using a workflow
		combining data from aerial photographs obtained from regularly scheduled surveys and field plot
		measurements from periodic forest inventories conducted in Bavarian state forests. Semi-Global
		Matching was used to derive surface models from the air photos which were normalized with terrain
		models from airborne laserscanning to derive canopy height models (CHM). Based on the CHM
		values at the respective field plots, a set of 14 predictor variables characterizing the height distribution
		was computed. For the prediction, individual Random Forests models were trained and cross
		validated separately for both test sites. With respect to relative RMSEs, i.e. divided by the observation
		means, most distinct differences were observed for the prediction of $QMD$ with a slightly higher level
		of accuracy for the spruce-dominated forest. Best results were achieved for $H_L$, while poorest model
		performances were obtained for $N$. The relative plot-level RMSEs for $N$, $G$, $QMD$, $V$, and $H_L$ were:
		$70.3\%$, $36.0\%$, $32.3\%$, $37.8\%$, and $12.4\%$ for the beech-dominated and $74.9\%$, $35.2\%$, $24.9\%$, $33.3\%$,
		and $12.4\%$ for the spruce-dominated forest. Thus, with the exception of $QMD$, forest type did not
		considerably influence the model accuracies.
	\item[Contribution] The idea for this comparative analysis of models for a set of forest inventory attributes 
		across two different forest types was developed together by both authors. 
	
		For the implementation, 
		\myName processed the aerial imagery and the \ac{SGM} data to generate the \acp{CHM} for both test sites.
		He also computed the not readily available forest inventory attributes based on the tree list data
		from the ground plot measurements. Christoph Straub calculated the set of predictor variables
		from the \ac{CHM} height values at the inventory plots and built the predictive \ac{RF} models 
		based on existing code developed in \textcite{Stepper.2015b}. 
	
		Both authors conducted the analysis and interpretation of the modelling results and wrote the manuscript.
\end{description}



\subsection[Stepper et al. \emph{Scand. J. For. Res.} (submitted)]
{Stepper et al.\; 
	\emph{Scand. J. For. Res.}
	(submitted) }
\label{sec:pub.ScandJForRes2016}

\begin{description}
	\item[Title] Using canopy heights from digital aerial photogrammetry to enable spatial transfer of forest attribute models: 
		a case study in central Europe
	\item[Authors] \emph{Christoph Stepper}, Christoph Straub, Markus Immitzer, and Hans Pretzsch
	\item[Journal] Scandinavian Journal of Forest Research (Publisher: Nordic Forest Research Cooperation Committee, 
		Taylor \& Francis, \textsc{issn}: 0282-7581, Journal Impact Factor, \emph{2014}: 1.537)
	\item[Abstract] This paper describes a workflow utilizing detailed canopy height information derived from digital airphotos
		 combined with ground inventory information gathered in state-owned forests and regression modelling techniques to 
		 quantify forest growing stocks in private woodlands, for which little information is generally available. 
		 Random forest models were trained to predict three different variables at the plot level: 
		 quadratic mean diameter of the 100 largest trees ($d_{100}$), basal area weighted mean height of the 100 largest trees ($h_{100}$), and gross volume (($V$). 
		 Two separate models were created -- one for a spruce- and one for a beech-dominated test site. 
		 We examined the spatial portability of the models by using them to predict the aforementioned variables at actual inventory plots in nearby forests, 
		 in which simultaneous ground sampling took place. When data from the full set of available plots were used for training, 
		 the predictions for $d_{100}$, $h_{100}$, and $V$ achieved out-of-bag model accuracies (scaled RMSEs) 
		 of $15.1\%$, $10.1\%$ and $35.3\%$ for the spruce- and $15.9\%$, $9.7\%$,
		 and $32.1\%$ for the beech-dominated forest, respectively.
		 The corresponding independent RMSEs for the nearby forests were $15.2\%$, $10.5\%$, and $33.6\%$ for the spruce-
		 and $15.5\%$, $8.9\%$, and $33.7\%$ for the beech-dominated test site, respectively.
%		Ground based inventory information on the publicly owned forests is most often
%		available from periodic field measurements. However, little information is
%		available for the private woodlands, particularly for small parcels of land as
%		common in e.g. Bavaria, Germany, and other European regions. Addressing the
%		lack of inventory data, this paper describes the development of a workflow
%		providing the missing link by means of remotely sensed auxiliary data. Canopy
%		metrics from image-based point clouds generated using semi-global matching of
%		air photos were combined with ground plot data, and random forests models were
%		trained to predict three different variables: quadratic mean diameter of the 100
%		largest trees ($d_{100}$), basal area weighted mean height of the 100 largest trees ($h_{100}$),
%		and gross volume ($V$) at the plot level. We trained separate models for a spruce
%		and a beech-dominated test site, with the aim to cover the most common forest
%		environments in Bavaria. We examined the spatial portability by predicting the
%		variables at inventory plots in close-by forests, in which simultaneous ground
%		sampling took place. Using the full set of available plots for training, the plot
%		level predictions for $d_{100}$, $h_{100}$, and $V$ achieved out-of-bag model accuracies
%		(scaled RMSEs) of $15.1\%$, $10.1\%$ and $35.3\%$ for the spruce- and $15.9\%$, $9.7\%$,
%		and $32.1\%$ for the beech-dominated forest, respectively. Corresponding
%		independent RMSEs for the predictions to the neighbouring forests were $15.2\%$,
%		$10.5\%$, and $33.6\%$ for the spruce- and $15.5\%$, $8.9\%$, and $33.7\%$ for the beech
%		dominated test site, respectively. Our findings indicate that area-covering canopy
%		heights obtained from digital aerial photogrammetry can be a viable source of
%		auxiliary data to unveil information for privately owned forests. Providing
%		spatially explicit, high-resolution inventory information on private estates
%		facilitates sustainable planning for the forest owners and helps the state forest
%		authority to formulate focused policies.
	\item[Contribution] \myName developed the idea for this study and the other authors contributed to the experimental layout. 
	
		\myName conducted the analysis. He processed the image data and prepared the inventory data for subsequent use in modelling. 
		He wrote the \textsf{R} code following previous work \parencite{Stepper.2015b} and performed all modelling steps. 
		\myName discussed the results, considering helpful comments of the co-authors. 
		
		\myName wrote the manuscript and compiled the figures, with comments and advise from all other authors. 
	
\end{description}



\subsection[White et al. \emph{Forests} (2015)]
{White et al.\; 
	\emph{Forests}
	(2015) }
\label{sec:pub.Forests2015}

\begin{description}
	\item[Title] Comparing ALS and image-based point cloud metrics and modelled forest inventory attributes
		in a complex coastal forest environment
	\item[Authors] Joanne C. White, \emph{Christoph Stepper}, Piotr Tompalski, Nicholas C. Coops, and Michael A. Wulder
	\item[Journal] Forests (Publisher: MDPI, \textsc{issn}: 1999-4907, Journal Impact Factor, \emph{2014}: 1.449)
	\item[Abstract] Digital aerial photogrammetry (DAP) is emerging as an alternate data source to
		airborne laser scanning (ALS) data for three-dimensional characterization of forest
		structure. In this study we compare point cloud metrics and plot-level model estimates
		derived from ALS data and an image-based point cloud generated using semi-global
		matching (SGM) for a complex, coastal forest in western Canada. Plot-level estimates of
		Lorey’s mean height ($H$), basal area ($G$), and gross volume ($V$) were modelled using an
		area-based approach. Metrics and model outcomes were evaluated across a series of strata
		defined by slope and canopy cover, as well as by image acquisition date. We found
		statistically significant differences between ALS and SGM metrics for all strata for five of
		the eight metrics we used for model development. We also found that the similarity
		between metrics from the two data sources generally increased with increasing canopy
		cover, particularly for upper canopy metrics, whereas trends across slope classes were less
		consistent. Model outcomes from ALS and SGM were comparable. We found the greatest 
		difference in model outcomes was for~$H$ ($\Delta RMSE = 5.04\%$). By comparison, $\Delta RMSE$ was 
		$2.33\%$ for $G$ and $3.63\%$ for $V$. We did not discern any corresponding trends in model
		outcomes across slope and canopy cover strata, or associated with different image
		acquisition dates.
	\item[Contribution] This research was made possible by international collaboration of the \myInstitute,
		the Department of Forest Research at University of British Columbia, and the Canadian Forest Service. 
		The contact was made by \myName and Joanne White. 
	
		Joanne White and \myName set up the design of the study, with conceptual advice of all other authors. 
		Joanne White coordinated the progress of the study, including the contacts with external partners for acquisition 
		of all field inventory and remote sensing data used. \myName conducted all photogrammetry work for obtaining image-based
		canopy height information. Piotr Tompalski and \myName processed the \ac{ALS} and \ac{DAP} data for further analysis.
		Joanne White lead investigations on the comparison of point cloud metrics across a series of strata defined by slope and canopy cover
		and Christoph Stepper was responsible for modelling and analysing plot level predictions for Lorey’s mean height,
		basal area, and gross volume. 
		All authors discussed the results and implications and contributed in formulating the findings of that research.
	
		Joanne White and \myName wrote the manuscript. \myName, Joanne White, and Piotr Tompalski prepared the figures and assembled the artwork.
		All authors commented on the manuscript and participated in revising the manuscript.
	
\end{description}



\subsection[Immitzer et al. \emph{Forest Ecol. Manag.} (2016)]
{Immitzer et al.\; 
	\emph{Forest Ecol. Manag.}
	(2016) }
\label{sec:pub.Forecol2016}

\begin{description}
	\item[Title] Use of WorldView-2 stereo imagery and National Forest Inventory data for wall-to-wall mapping of growing stock
	\item[Authors] Markus Immitzer, \emph{Christoph Stepper}, Sebastian Böck, Christoph Straub, and Clement Atzberger
	\item[Journal] Forest Ecology and Management (Publisher: Elsevier, \textsc{issn}: 0378-1127, 
		Journal Impact Factor, \emph{2014}: 2.660)
	\item[Abstract] Angle-count sampling (ACS) is an established method in forest mensuration and is implemented in different
		National Forest Inventories (NFI). However, due to the lack of fixed reference areas of the inventory
		plots, these ACS-based field data are seldom used as training data for wall-to-wall mapping applications
		at forest enterprise level. In this paper, we demonstrate an approach to overcome this shortcoming. For a
		study area in northern Bavaria, Germany, we used ACS-based NFI data for model training to generate
		wall-to-wall maps of growing stock for broadleaf, conifer and mixed forest stands. Both spectral and
		height information from the very high resolution World\mbox{View-2} (WV2) satellite were used as auxiliary
		information and the non-parametric Random Forests (RF) algorithm was chosen as modeling approach.
		The growing stock predictions were validated using out-of-bag (OOB) samples and further verified at
		the plot and stand level using additional data. For validation, field plots from a Management Forest
		Inventory (MFI) and delineated forest stands were used. Compared to stand-level aggregations based
		on field plots from the MFI, our approach explained $56\%$ of the variability in the growing stock ($R^2$) with
		a relative RMSE of $15\%$ at the stand level ($n = 252$). As expected, the scatter was higher at the plot-level
		($n = 3973$). Nonetheless, the models still achieved acceptable performance measures ($R^2 = 0.44$;
		$RMSE = 34\%$).
	\item[Contribution] Markus Immitzer and Clement Atzberger initiated the study. Markus Immitzer, Clement Atzberger, 
		and \myName contributed to the initial experimental design and developed the new approach for using \ac{ACS}-based \ac{NFI} plot data
		as training data for wall-to-wall mapping applications. The novelty of the approach, \ie, the simultaneous use of
		multiple concentric circles around the \ac{NFI} plot centres for calculating explanatory variables based on remote sensing information,
		was developed by Markus Immitzer and Clement Atzberger in close collaboration with \myName.
		
		Markus Immitzer processed the remote sensing data and conducted the main parts of the analysis,
		including programming of the procedures and compiling the results in tables and figures.
		Christoph Straub computed the vegetation mask based on the \ac{WV2} data.
		\myName prepared the inventory data used for model training (\acs{NFI}) and model evaluation (\acs{MFI}). 
		Sebastian Böck helped in programming and analysis.
	
		Markus Immitzer and \myName prepared all figures and flowcharts for the article and 
		wrote the initial draft of the manuscript together with Clement Atzberger.
		All authors contributed to the final manuscript and participated in revising the manuscript for publication.
	
\end{description}


\subsection[Stepper et al. \emph{Forestry} (2015)]
{Stepper et al.\; 
	\emph{Forestry}
	(2015) }
\label{sec:pub.Forestry2015}

\begin{description}
	\item[Title] Assessing height changes in a highly structured forest using regularly acquired aerial image data 
	\item[Authors] \emph{Christoph Stepper}, Christoph Straub, and Hans Pretzsch
	\item[Journal] Forestry (Publisher: Institute of Chartered Foresters, Oxford University Press, \textsc{issn}: 0015-752X, 
		Journal Impact Factor, \emph{2014}: 2.093)
	\item[Abstract] In this paper,we demonstrate the effectiveness of digital stereo images and canopy height models (CHMs) derived
		from them for forest height change assessment. Top heights were derived for 199 terrestrial inventory plots from
		forest inventories conducted in 2008 and 2013 in a forest near Traunstein, Germany. Semi-Global Matching was
		applied to two sets of aerial stereo images, acquired in 2009 and 2012, respectively, to compute CHMs. Subsequently,
		several height percentiles were calculated from the areas in the CHMs that lay within the inventory plot
		locations. The maximum CHM value ($h_{max}$) had the highest correlation with the field-based canopy top heights
		and was selected for use in all further analysis. Periodic annual increments (PAIs) of forest height were calculated
		from both the remote sensing and the field data at the inventory plot locations. Scatterplots of the PAIs over top
		height revealed similar patterns in the results derived from the two data sets. The inventory plots were assigned to
		three height classes representing various forest successional stages --- \emph{youth}, \emph{full vigour} and \emph{old age}.
		The PAI distributions
		within the three height classes were significantly different from one another. Our findings suggest
		that CHMs derived from repeat aerial image surveys can be a viable tool to measure canopy heights and to
		assess forest height changes over time, even for a highly structured, mixed forest in central Europe.
	\item[Contribution] All authors contributed to the idea of the work and jointly conceived the experiment.
	
		\myName made all data ready for analysis, including the processing of the aerial imagery and the ground inventory data. 
		He conducted the main parts of the statistical analysis, \ie, selecting the appropriate methods,
		writing the code and carrying out the experiments. \myName and Christoph Straub examined and discussed the results.
		Hans Pretzsch contributed in result interpretation and in placing the findings into a broader context in the forest research discipline.
	
		\myName wrote the first draft of the manuscript and prepared all figures and artwork, 
		with valuable contributions of Christoph Straub. 
		Hans Pretzsch added helpful edits to the manuscript and gave advice on the content. 
	
\end{description}

