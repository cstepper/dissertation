%************************************************
\chapter{Discussion}
\label{chp:Discussion}
%************************************************

% This is a combining discussion for the different aspects presented in the scientific papers above.

Different aspects of utilizing digital airphotos and height measurements derived thereof by means of \ac{DAP} 
for forest inventory and forest management were considered in the studies presented in this thesis.
Overall, the results confirm that image-based point clouds and \acp{CHM} are a viable source of information
to characterize the three-dimensional structure of forest canopies, at least for the forest types examined
in the different studies (see Table~\ref{tab:TestSites}).

The primary motivations for initiating this work were:
\begin{itemize}
%	\item the growing interest of forest professionals in the use of high spatial resolution \ac{DAP} to generate 
%		3d information to support forest inventory and monitoring \parencite{White.2016}.
	\item the need for spatially explicit information on the current state of forest stands in order to facilitate forest management practice;
	\item the unexploited potential of using digital aerial photographs for the three-dimensional characterization of forest canopies;
	\item the recently launched algorithms to process stereo airphotos to generate image-based height measurements covering large areas;
	\item the possibility to test the \ac{DAP}-based height measurement as surrogate for \ac{ALS}-based data when spatially predicting forest inventory attributes.  
\end{itemize} 


As mentioned in the introduction, large parts of the theoretical framework on using 3d information in forest inventory
were developed based on \ac{ALS} data (\eg, \cite{Naesset.2002b}).
Meanwhile, airborne imaging technologies (\eg, digital cameras) and image processing algorithms and software (\eg, \ac{SGM}) 
have emerged that enable workflows possible to generate image-based point clouds and \acp{DSM} covering large areas \parencite{White.2016}.
Those 3d measurements achieve similar degrees of detail for forest canopy surfaces as the airborne \ac{LiDAR}-based measurements,
and with the large-scale availability of \ac{ALS}-based \acp{DTM}, canopy heights above ground can be derived from 
the photogrammetric data with little additional effort. 

\medskip

\begin{itemize}
	\item Schwerpunkt der Studien: Mitteleuropäische Waldverhältnisse --> Ergebnisse für Schätzung wichtiger Forstlicher Kennzahlen ok
	\item Vergleich mit LiDAR in Canada --> DAP Daten sind ein sinnvolle Ergänzung zu Laser Scanning Daten --> Ermöglichen neue Inventurfortschreibungsverfahren
	 --> viel billiger als Lidar \\ --> regelmäßige Befliegung \\ --> zusätzliche Verfügbarkeit der Spektralen Information
\end{itemize}

\medskip
\begin{itemize}
	\item Wie genau lassen sich die verschiedenen forstlichen Parameter mit der gewählten Methode abschätzen?
	\item auf die unterschiedlichen Kennzahlen eingehen. 
	\item Holzvolumen --> nächste Schritte: Einzelbaum-Verfahren / Durchmesserverteilung approximieren
	\item stärkere Integration der spektralen Kennwerte in die Modellierung (evtl. auch Verwendung von hochaufgelösten Satellitendaten (Phänologie), um Tree species unterscheidbar zu machen)
	\item was waren die wichtigsten Prädiktoren für die unterschiedlichen forstlichen Parameter --> Vergleich der Studien --> allgemeingültige Aussage möglich? Tendenzen? Vergleich mit Lidar?
	\item untersucht: Wie verhalten sich die Modelle in nadel- bzw. laubdominierten Wäldern! Frankenwald vs. Spessart/Steigerwald // Mischwald (Wald der Zukunft) --> Traunstein?
	\item ungelöste Aufgaben: Aussagen über die Baumarten-spezifischen Angaben (z.b. Holzvorrat Nadelbäume vs. Holzvorrat Laubbäume)
 	
\end{itemize}










% a now and emerging area of research is the generation and exploration of 3d information from airborne imagery,
% which has been enabled by recent advances in cameras and computing technologies.


